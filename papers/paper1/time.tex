\documentclass[11pt]{article}
\usepackage{amsmath, amssymb, hyperref}
\usepackage{amsthm}
\newtheorem{lemma}{Lemma}
\usepackage{geometry}
\geometry{margin=1in}
\title{Time as an Emergent Property of the Observer: A Derivation from Axiomatic Simulation}
\author{Juha Meskanen}
\date{July 2019}

\begin{document}

\maketitle

\begin{abstract}
  We argue that a human being, including consciousness, must be an axiomatic system if four reasonable
  assumptions hold: (1) DNA contains the information required to construct a conscious observer;
  (2) DNA obeys the known laws of physics; (3) the Church--Turing thesis is valid; and (4) pain has measurable effects.
  From these assumptions, we derive that time is not a fundamental property of the universe but a feature of the
  observer's internal information structure.
\end{abstract}

\section{Introduction}

The laws of physics are mathematical in nature, and mathematics is based on axiomatic systems.
If DNA encodes the complete blueprint for constructing a conscious observer, and if DNA operates according to physical laws,
then humans themselves must be instances of axiomatic systems.
If the Church--Turing thesis holds, then any process governed by physical law can be simulated by a Turing machine.

We now introduce a fourth assumption:

\begin{itemize}
  \item \textbf{Pain Has Measurable Effects:} Pain produces measurable physical outcomes that can be externally detected.
\end{itemize}

Two identical axiomatic systems, regardless of their physical substrate, must exhibit identical behavior. It follows that a Turing-equivalent simulation of DNA must also give rise to a conscious observer, provided the assumptions hold.

This leads us to ask: what is the nature of time in such a system?

\section{DNA Simulation Thought Experiments}

\subsection{Single-threaded Simulation}

Suppose we digitize a human genome and run it in a silicon-based computer simulating a universe governed by the same laws of physics as our own. As the simulation executes, the DNA unfolds and—by assumption—evolves into a conscious, pain-sensitive observer. This simulated observer may reflect on its own experience of time.

Now imagine we optimize the simulation. We replace algorithmic components (e.g., \texttt{sqrt()}) with lookup tables, compile to efficient machine code, or precompute results. As optimizations accumulate, CPU cycles decrease.

In an axiomatic system, such optimizations do not alter the system's internal structure or its outcomes. Gravity behaves identically whether computed via numerical algorithms or retrieved from tables. The same applies to a DNA simulation: the observer’s experience should remain invariant under implementation changes. 2+2 is always 4, whether calculated or looked up.

Consider the extreme case: we replace all computation with a static dataset encoding the entire execution trace. If the structure remains intact, is the observer still conscious?

Yes—because to claim otherwise would require positing a physical threshold (such as a minimum number of CPU cycles per bit of information) for consciousness, which contradicts the axiomatic model.

\textbf{Conclusion:} If an execution trace encodes a conscious experience—including pain, memory, and time perception—then that experience exists within the trace itself, regardless of whether it was ever executed. Time must therefore be a property of the information structure, not of its runtime.

\subsection{Multithreading and White Noise}

Now consider a system running multiple DNA simulations concurrently—say, Alice and Bob—where quantum randomness drives the switching between simulations. The resulting trace interleaves their lives in unpredictable segments. Yet each observer experiences a coherent, continuous timeline.

As the number of concurrent simulations increases, the trace becomes increasingly fragmented. In the limit of infinite simulations with perfect interleaving, the trace approaches pure white noise. Yet, from the internal perspective of each observer, consciousness persists.

\textbf{Conclusion:} Consciousness can emerge from static bitstrings that externally resemble white noise. Continuity of experience is defined internally by the observer’s information coherence, not by the order or contiguity of global execution.

\subsection{Observer Continuity in Interleaved Execution Traces}

In a multithreaded simulation where multiple DNA-based agents are executed concurrently, the global trace may appear chaotic—interleaving fragments of Alice, Bob, and others in a seemingly random sequence. As more simulations are added and switching is randomized, the trace increasingly resembles white noise.

Yet each observer experiences a coherent and consistent life. We know from practical computing that multithreaded programs execute correctly: simulation results are invariant whether threads are interleaved or serialized.

\begin{quote}
  How can an observer maintain subjective continuity when their trace is embedded in an interleaved, noise-like execution stream?
\end{quote}

We answer this with the following result:

\begin{lemma}[Observer Continuity from Information Consistency]
  Let $T$ be an execution trace composed of interleaved fragments from multiple simulated observers.
  If a subset $T_i \subset T$ forms a causally consistent information structure with persistent internal memory
  and lawful state evolution, then $T_i$ corresponds to a valid observer timeline. All unrelated fragments are ignored.
\end{lemma}

\begin{proof}
  Consciousness, by assumption, is a product of causal information processing governed by physical laws
  and DNA-derived rules. If a subset of the trace:
  \begin{itemize}
    \item Preserves self-identity through internal memory,
    \item Evolves according to causal laws,
    \item Produces behavior consistent with observable outputs (e.g., reactions to pain),
  \end{itemize}
  then it satisfies the criteria for a conscious observer.

  Interleaved execution does not destroy this internal causal structure—it merely disperses it. The observer emerges by integrating its own causal thread, filtering out unrelated noise. The observer is the attractor: it selects and organizes events that maintain internal coherence.

  All other fragments in $T$ lack such continuity relative to $T_i$ and are therefore excluded from the observer’s timeline.

  Thus, the observer is not defined externally by execution order, but internally by information consistency.
\end{proof}

\textbf{Conclusion:} Consciousness depends on causal coherence, not on temporal or spatial contiguity. Time, identity, and experience arise from the observer’s ability to construct a self-consistent narrative from the trace—even when the global trace resembles noise.

\subsection{Causality and Substrate Equivalence}

It is often assumed that executing a simulation causes a virtual universe to “come into existence”—that a computer must be powered on for a simulated being to feel pain. But the above thought experiments show that the complete conscious experience already exists within the static execution trace.

Because this trace has no causal dependence on the physical substrate that encodes or executes it, the relationship between hardware and simulation is representational, not causal.

\begin{quote}
  \textbf{Lemma (Time is Observer-Relative):} \emph{Time is not a fundamental property of the universe, but an emergent property of the observer’s internal information state.}
\end{quote}

\textbf{Final Insight:} Just as a black hole's collapse and an execution trace of vanishing entropy are dual descriptions of the same phenomenon, so too are a running simulation and the virtual universe it encodes.

The executing machine and the observer experiencing pain are simply two arrangements of the same information.

This mirrors earlier results connecting gravity, entropy, and geometric singularities: the physical and informational descriptions are two complementary views of the same underlying structure.

\end{document}

