\documentclass[11pt]{article}
\usepackage{amsmath, amssymb, hyperref}
\usepackage{amsthm}
\newtheorem{lemma}{Lemma}
\usepackage{geometry}
\geometry{margin=1in}
\title{Time as an Emergent Property of the Observer: A Derivation from Axiomatic Simulation}
\author{Juha Meskanen}
\date{July 2019}

\begin{document}

\maketitle

\begin{abstract}
  We argue that a human being, including consciousness, must be an axiomatic system if four reasonable
  assumptions hold. From these axioms, we proof that time is not a fundamental property of the universe but a feature of the
  observer's internal information structure.
\end{abstract}

\section{Introduction}

Physics has not yet answered the question of how time and consciousness
relate to the fundamental structure of reality. However, based on the recent advances in science and technology, DNA seems to contain
information of how to construct conscious observer in this universe.

Let us make the following three assumptions:

\begin{itemize}
  \item \textbf{DNA Contains Information for Consciousness:} DNA encodes the necessary information to construct an observer capable to feel pain.
  \item \textbf{DNA Obeys Physical Laws:} DNA operates according to the laws of physics.
  \item  \textbf{Church--Turing thesis holds:}  Any process governed by physical law can be simulated by a Turing machine.
\end{itemize}

It follows that humans themselves must be instances of axiomatic systems, and as such can be duplicated on an implementation
of axiomatic system, such as a Turing machine, or its moderl equivalent - computer.


We now introduce a fourth assumption that translates subjective human experience from philosophical abstraction to measurable physical phenomena:

\begin{itemize}
  \item \textbf{Pain Has Measurable Effects:} Pain produces measurable physical outcomes that can be externally detected and measured.
\end{itemize}

We can formally express this:

\begin{equation}
  A + \text{Pain} \neq A
\end{equation}

Two identical axiomatic systems, regardless of their physical substrate, must exhibit identical behavior. It then follows that those simulated copies must feel pain exactly as we humans feel it.
Otherwise there would be a contradiction.



\section{DNA Simulation Thought Experiments}

On the contrary what the current physical theories suggest, we experience time as one-way flow, from past to future.
Simulated observers must experience time indistinguishable from real observers, because they are instances of the same
axiomatic system as we humans are.

We can now ask: what is it that the simulated human experience as time?

All the information in the simulated universe is defined by the Turing Machine running the simulation. And we can fully understand the operation of Turing Machine.


\subsection{Single-threaded Simulation}

Suppose we digitize a human genome and run it in a silicon-based computer simulating a universe governed by the same laws of physics as our own.
As the simulation executes, the DNA evolves into a conscious, pain-sensitive observer, who experience living in an universe where time flows from past to future.

Now imagine we gradually optimize the simulation. We replace algorithmic components (e.g., \texttt{sqrt()}) with lookup tables, compile to efficient machine
code, or precompute results. As optimizations accumulate, CPU cycles decrease.

How does this affect the observer's experience of time? Answer: it does not. Such optimizations do not alter the system's internal structure
or its outcomes. The observer’s time experience must remain invariant under implementation changes.

Consider the extreme case: we replace all computation with a static dataset encoding the entire execution trace - a long sequence of bits.
If the structure remains intact, is the observer still conscious?

Yes—because to claim otherwise would require positing a physical threshold (such as a minimum number of CPU cycles per bit of information) for consciousness, which contradicts the axiomatic model.

\textbf{Conclusion:} If an execution trace of the computer encodes a conscious experience—including pain, memory, and time perception—then that experience exists within the trace itself,
regardless of whether it was ever executed. Time must therefore be a property of the information structure, not of its runtime.


\subsection{Multithreading and White Noise}

Now consider a system running multiple DNA simulations concurrently—say, Alice and Bob—where quantum randomness drives the switching between simulations.
The resulting execution trace interleaves their lives in  segments with unpredictable length. Yet we know both single and multi-threaded computers work equally well.
Correspondingly, each observer experiences a coherent, continuous timeline.

As the number of concurrent simulations increases, the trace becomes increasingly fragmented. In the limit of infinite simulations with perfect interleaving, the trace
approaches pure white noise. Yet, from the internal perspective of each observer, time flows from past to future, and pain must feel precisely the same.

\textbf{Conclusion:} Consciousness in the simulated universe emerges from static information that resemble white noise. Continuity of experience is defined internally by the observer’s information
coherence, not by the order or contiguity of global execution.


\subsection{Observer Continuity in Interleaved Execution Traces}

In a multithreaded simulation where multiple DNA-based agents are executed concurrently, the global trace may appear chaotic—interleaving fragments of Alice, Bob, and others in a seemingly random sequence. As more simulations are added and switching is randomized, the trace increasingly resembles white noise.

Yet each observer experiences a coherent and consistent life.
\begin{quote}
  How can an observer maintain subjective continuity when their trace is embedded in an interleaved, noise-like execution stream?
\end{quote}

We answer this with the following result:

\begin{lemma}[Observer Continuity from Information Consistency]
  Let $T$ be an execution trace composed of interleaved fragments from multiple simulated observers.
  If a subset $T_i \subset T$ forms a causally consistent information structure with persistent internal memory
  and lawful state evolution, then $T_i$ corresponds to a valid observer timeline. All unrelated fragments are ignored.
\end{lemma}

\begin{proof}
  Consciousness, by assumption, is a product of causal information processing governed by physical laws
  and DNA-derived rules. If a subset of the trace:
  \begin{itemize}
    \item Preserves self-identity through internal memory,
    \item Evolves according to causal laws,
    \item Produces behavior consistent with observable outputs (e.g., reactions to pain),
  \end{itemize}
  then it satisfies the criteria for a conscious observer.

  Interleaved execution does not destroy this internal causal structure—it merely disperses it. The observer emerges by integrating its own causal thread, filtering out unrelated noise.
  The observer is the attractor: it selects and organizes events that maintain internal coherence.

  All other fragments in $T$ lack such continuity relative to $T_i$ and are therefore excluded from the observer’s timeline.

  Thus, the observer is not defined externally by execution order, but internally by information consistency.
\end{proof}

\textbf{Conclusion:} Consciousness depends on causal coherence, not on temporal or spatial contiguity. Time, identity, and experience arise from the observer’s ability to construct a self-consistent narrative from the trace—even when the global trace resembles noise.

\subsection{Causality and Substrate Equivalence}

It is often assumed that executing a simulation causes a virtual universe to “come into existence”—that a computer must be powered on for a simulated being to feel pain.
But the above thought experiments show that the complete conscious experience already exists within the static execution trace.

Because this trace has no causal dependence on the physical substrate that encodes or executes it, the relationship between hardware and simulation is representational, not causal.

\begin{quote}
  \textbf{Lemma (Time is Observer-Relative):} \emph{Time is not a fundamental property of the universe, but an emergent property of the observer’s internal information state.}
\end{quote}

The executing machine and the observer experiencing pain are two arrangements of the same information. Such information can appear pure noise to us, but appear as conscious and pain living in expanding universe where time flows from past to future.


\textbf{The Nature of the Universe}

A measurement a simulated observer makes in their simulated universecorrespond to the same measurement a real observer in a real world carries out. The virtual observer feels the virtual universe real.

However, we know that those virtual universes are not real, because we created them using Turing Machine, whose operation we can understand.

We model the universe recursively:

\[
  U_{i+1} = U_i
\]

where \( U_i \) is the real universe and \( U_{i+1} \) is the virtual universe. The recursive relation implies that the universe is self-similar at all levels of abstraction.

Assume there exists a subset \( S \subseteq U_i \) such that:

\[
  S_{i+1} + \text{Pain} = S_i + \text{Pain}
\]

Now, if:

\[
  U_{i+1} = \text{Virtual}
\]

Then, by the recursive relation:

\[
  U_i = U_{i+1} = \text{Virtual}
\]

\textbf{Conclusion:} We are living in a virtual universe.


\textbf{Conclusions:}

The above thought experiments lead us to conclude that:
\begin{itemize}
  \item Time is not a fundamental property of the universe, but an emergent property of the observer's internal information structure.
  \item Consciousness and pain are properties of information structures, not physical substrates.
  \item The deep nature of the universe is fundamentally abstract information.
\end{itemize}


\section{Future Work}

Future work will explore the implications of this model for understanding the physics as emergent properties, and how physical phenomeon could be represented as pure informational structures.

\end{document}

