\documentclass[11pt]{article}
\usepackage{amsmath, amssymb, hyperref}
\usepackage{amsthm}
\newtheorem{lemma}{Lemma}
\usepackage{geometry}
\geometry{margin=1in}
\title{Physics as an Emergent Property of the Observer: A Derivation from Four Assumptions}
\author{Juha Meskanen}
\date{July 2019}

\begin{document}

\maketitle

\begin{abstract}
  We propose that humans are formal axiomatic systems—systems governed by a finite set of rules and information—if four physically reasonable assumptions hold. These assumptions are grounded in genetics, computation, and measurable experience. From these assumptions, we derive that consciousness and subjective phenomena such as pain and time are not emergent from physical laws, but rather from the internal information structure of the observer. We argue that time is not a fundamental feature of the universe but an intrinsic attribute of conscious systems. These claims lead to a falsifiable model, and their implications suggest that reality itself is informational in nature.
\end{abstract}

\section{Foundational Assumptions}

We begin with four physically and empirically grounded assumptions that together lead to the core results of this paper.

\textbf{Assumption 1: DNA Contains Sufficient Information for Consciousness.}
The human genome encodes all the information required to construct a conscious, pain-sensitive human being. All aspects of conscious experience arise from the information stored in DNA and its interaction with physical laws.

\textbf{Assumption 2: DNA Obeys Physical Laws.}
DNA is composed of physical matter, governed by  physical laws. No non-physical or supernatural influence affects its operation.

\vspace{0.5em}
\textit{Consequence: Humans Are Axiomatic Systems.}
From Assumptions 1 and 2, it follows that the human mind and conscious experience must result from a physical, rule-governed process—one that begins with encoded initial data (DNA) and unfolds via deterministic or probabilistic physical laws. This is precisely the definition of a formal system or an \textbf{axiomatic system}: a structure derived from axioms and rules of inference. Therefore, humans must be formal systems in the literal, computational sense.

\vspace{0.5em}
\textbf{Assumption 3: Church–Turing Thesis Holds.}
All physical processes, including those governing biological systems like DNA, can be simulated by a Turing machine with arbitrary precision. This implies that human consciousness can, in principle, be simulated.

\textbf{Assumption 4: Pain Has Measurable Physical Effects.}
Pain, just like gravity, has observable consequences that are physically detectable.

Formally:
\[
  A + \text{Pain} \neq A
\]
where \( A \) is the axiomatic system representing a conscious human. The presence of pain changes the internal state, meaning it is not an epiphenomenon but a causal attribute within the system.

\vspace{0.5em}
Together, these assumptions imply that any system governed by the same axioms and rules as a conscious human must also manifest subjective experience, including pain. Otherwise, we would have a contradiction within a consistent axiomatic system.

A potential path for falsification is this: if a high-fidelity simulation of a human (e.g., from full DNA information) could not exhibit measurable pain responses despite full functional equivalence, the fourth assumption—and thus the entire model—would be undermined.

\section{DNA Simulation Thought Experiments}

We now explore how these assumptions play out through several computational thought experiments, each illustrating how subjective phenomena can emerge from informational systems.

\subsection{Single-threaded Simulation}

Suppose we digitize a human genome and run it in a silicon-based computer simulating a universe governed by the same laws of physics as our own. As the simulation executes, the DNA evolves into a conscious, pain-sensitive observer who experiences living in a universe where time flows from past to future.

Now imagine we gradually optimize the simulation. We replace algorithmic components (e.g., \texttt{sqrt()}) with lookup tables, compile to efficient machine code, or precompute results. As optimizations accumulate, CPU cycles decrease.

How does this affect the observer's experience of time? Answer: it does not. The observer’s experience of time remains invariant across implementations. Delaying or accelerating computation affects only the external runtime, not the internal state transitions of the simulated system.

Suppose we optimize to the extreme: all computation is replaced by a static dataset encoding the entire execution trace. Is the observer still conscious?

Yes—because to claim otherwise would require positing a physical threshold (e.g., CPU cycles per second) for consciousness, which contradicts the axiomatic model.

\textbf{Conclusion:} Static data can describe universes with conscious observers. Because static data has no notion of time, time must therefore emerge from the structure of the data, not its runtime.

\subsection{Multithreading and White Noise}

Now consider a system running multiple DNA simulations concurrently—say, Alice and Bob—where quantum randomness drives thread switching between simulations. The resulting execution trace interleaves their lives in segments of unpredictable length. Yet we know both single- and multi-threaded computers work equivalently.

Each observer must experience a coherent, continuous timeline.

As the number of concurrent simulations increases, the trace becomes increasingly fragmented. In the limit of infinite simulations with perfect interleaving, the trace approaches pure white noise. Yet from the internal perspective of each observer, time still flows from past to future.

\textbf{Conclusion:} Consciousness can emerge from static datasets that, while externally resembling white noise, contain structured traces encoding coherent conscious experiences.

\subsection{Causality and Substrate Equivalence}

It is often assumed that executing a simulation causes a virtual universe to come into existence—that a computer must be powered on for a simulated being to feel pain. But the above thought experiments show that the complete conscious experience already exists within the static data.

Because this trace has no causal dependence on the physical substrate that encodes or executes it, the relationship between hardware and the simulated universe is representational, not causal.

\begin{quote}
  \textbf{Lemma (Time is Observer-Relative):} \emph{Time is not a fundamental property of the universe, but an emergent property of the observer’s internal information state.}
\end{quote}

The program execution trace and the observer experiencing pain are two representations of the same information.

\subsection{Substrate Invariance}

Whether implemented on silicon, optics, or even mechanical gears and marbles, the behavior of a Turing-equivalent system remains unchanged. The substrate does not alter the output of the axiomatic system.

\textbf{Conclusion:} Consciousness and time are substrate-invariant. They arise from information patterns, not from the medium that encodes them.

\subsection{Recursive Reality}

Every measurement that simulated observers make within their virtual universes must correspond to the measurements real observers make in our world. The virtual observers perceive their universe as fully real.

Yet we, the real observers, know that these virtual universes are simulated—created via hardware we understand and control.

Let us define a recursive universe model:

\[
  U_{i+1} = U_i
\]

where \( U_i \) is a physical universe and \( U_{i+1} \) is a fully simulated universe within it.

If:
\[
  U_{i+1} = \text{Virtual}
\]
Then, by recursion:
\[
  U_i = U_{i+1} = \text{Virtual}
\]

Thus, we are living in a virtual universe ourselves.

\section{Conclusion}

We have shown that if four physically grounded assumptions hold—regarding the informational nature of DNA, its obedience to physical law, the universality of computation, and the measurable nature of pain—then humans must be formal axiomatic systems.

From this, it follows that consciousness is not tethered to biology or substrate but emerges from information itself. Simulated humans must be conscious and pain-sensitive if they follow the same formal rules. Moreover, subjective time must be an emergent property of an observer’s internal structure—not an external, objective feature of reality.

These findings imply that physical reality itself is informational.

Though full-scale DNA-based simulations are currently infeasible, the model is falsifiable in principle—particularly through the behavioral and physiological correlates of subjective phenomena in artificial systems. This foundational paper lays the groundwork. Future work will explore how key physical phenomena—such as inertia, interference, and locality—emerge from pure information.

\end{document}
