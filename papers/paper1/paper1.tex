\documentclass[11pt]{article}
\usepackage{amsmath, amssymb, hyperref}
\usepackage{amsthm}
\usepackage[backend=biber]{biblatex}
\addbibresource{../references.bib}
\newtheorem{lemma}{Lemma}
\usepackage{geometry}
\geometry{margin=1in}
\title{Physics as an Emergent Property of the Observer: A Derivation from Four Assumptions}
\author{Juha Meskanen}
\date{July 2019}

\begin{document}

\maketitle

\begin{abstract}
  We propose that humans are formal axiomatic systems—in the mathematical sense—if four physically reasonable assumptions hold. These assumptions are grounded in genetics, computation, and measurable experience. From these assumptions, we derive that consciousness and subjective phenomena such as pain and time are not emergent from physical laws, but rather from the internal information structure of the observer. We argue that time is not a fundamental feature of the universe but an intrinsic attribute of conscious systems. Their implications suggest that reality itself is informational in nature. These claims lead to a falsifiable model.
\end{abstract}

\section{Foundational Assumptions}

We begin with four physically and empirically grounded assumptions that together lead to the core results of this paper.

\textbf{Assumption 1: DNA Contains Sufficient Information for Consciousness.}
The human genome encodes all the information required to construct a conscious, pain-sensitive human being. All aspects of conscious experience arise from the information stored in DNA and its interaction with physical laws.

\textbf{Assumption 2: DNA Obeys Physical Laws.}
DNA is composed of physical matter, governed by  physical laws. No non-physical or supernatural influence affects its operation.

\vspace{0.5em}
\textit{Consequence: Humans Are Axiomatic Systems.}
From Assumptions 1 and 2, it follows that the human mind and conscious experience must result from a physical, rule-governed process—one that begins with encoded initial data (DNA) and unfolds via deterministic or probabilistic physical laws. This is precisely the definition of a formal system or an \textbf{axiomatic system}.

\vspace{0.5em}
\textbf{Assumption 3: Church–Turing Thesis Holds.}
All physical processes, including those governing biological systems like DNA, can be simulated by a Turing machine with arbitrary precision. This implies that human consciousness can, in principle, be simulated.

\textbf{Assumption 4: Pain Has Measurable Physical Effects.}
Pain, just like gravity, has observable consequences that are physically detectable.

Formally:
\[
  A + \text{Pain} \neq A
\]
where \( A \) is the axiomatic system representing a conscious human. The presence of pain changes the internal state, meaning it is not an epiphenomenon but a causal attribute within the system.


\section{DNA Simulation Thought Experiments}

We now explore how these assumptions play out through several computational thought experiments, each illustrating how subjective phenomena can emerge from informational systems.

\subsection{Single-threaded Simulation}

Suppose we digitize a human genome and run it in a silicon-based computer simulating a universe governed by the same laws of physics as our own. As the simulation executes, the DNA evolves into a conscious, pain-sensitive observer who experiences living in a universe where time flows from past to future.

Now imagine we gradually optimize the simulation. We replace algorithmic components (e.g., \texttt{sqrt()}) with lookup tables, compile to efficient machine code, or precompute results. As optimizations accumulate, CPU cycles decrease.

How does this affect the observer's experience of time? Answer: it does not. The observer’s experience of time remains invariant across implementations. Delaying or accelerating computation affects only the external runtime, not the internal state transitions of the simulated system.

Suppose we optimize to the extreme: all computation is replaced by a static dataset encoding the entire execution trace. Is the observer still conscious?

Yes—because to claim otherwise would require positing a physical threshold (e.g., CPU cycles per second) for consciousness, which contradicts the axiomatic model.

Static data has no notion of time. Because execution ordering (not clock time) defines state transitions, temporal structure must arise from internal data relationships, not from external runtime.

\textbf{Conclusion:} Static data can describe universes with conscious observers.



\subsection{Multithreading and White Noise}

Now consider a system running multiple DNA simulations concurrently—say, Alice and Bob—where quantum randomness drives thread switching between simulations. The resulting execution trace interleaves their lives in segments of unpredictable length. We know both single- and multi-threaded computers work equivalently, each observer must experience a coherent, continuous timeline.

Now consider a scenario where CPU executes the code one bit at a  time, and each thread is limited to a single CPU cycle before switching. As the number of concurrent simulations increases, the execution trace becomes increasingly fragmented. In the limit of infinitely many simulations with perfect interleaving, the trace approaches pure white noise. Yet from the internal perspective of each observer, time still flows from past to future. Empirically, we know that multithreaded computers function reliably regardless of how few CPU cycles are allocated per thread context switch, or what is the size of CPU's internal registers.

\textbf{Conclusion:} Consciousness can emerge from data of pure static noise.


\subsection{Substrate Invariance}

Simulations can be implemented in a variety of ways to execute mathematical algorithms.

The previously discussed static data thought experiments reflect common optimization techniques used to accelerate execution and reduce the number of steps required to run a simulation. However, such static data does not arise spontaneously; it must be precomputed—potentially by a human using pencil and paper. Central processing units (CPUs) may also be constructed from optical switches, or even mechanical systems such as wooden gates and spheres. Regardless of whether it is implemented in silicon, optics, or mechanical systems such as gears and marbles, the behavior of a Turing-equivalent system remains invariant.

\textbf{Conclusion}: The substrate does not affect the output of the axiomatic system.



\subsection{Causality and Substrate Equivalence}

It is often assumed that executing a simulation causes a virtual universe to come into existence—that a computer must be powered on for a simulated being to feel pain. But the above thought experiments show that the complete conscious experience already exists within the static data.

Because this trace has no causal dependence on the physical substrate that encodes or executes it, the relationship between hardware and the simulated universe is representational, not causal.

\begin{quote}
  \textbf{Lemma (Time is Observer-Relative):} \emph{Time is not a fundamental property of the universe, but an emergent property of the observer’s internal information state.}
\end{quote}

The program execution trace and the observer experiencing pain are two representations of the same information.

\textbf{Conclusion:} Consciousness and the subjective experience of time are substrate-invariant. They arise from informational structures rather than from the physical medium that encodes them.



\subsection{Information Space}

There is more than just two above described configurations - hardware and the simulation - in which information can be transformed. It can be permuted in $2^n$ distinct ways for an $n$-bit sequence. Among these permutations, one or more configurations may correspond to conscious processes.

Thus, the only formal constraint for the potential emergence of consciousness is the size of the information space:

\[
  \text{Consciousness} \in \mathcal{I}_n \quad \text{if and only if} \quad n \geq n_{\text{min}}
\]

where $\mathcal{I}_n$ denotes the set of all possible $n$-bit information patterns, and $n_{\text{min}}$ represents the minimum number of bits required to support the complexity of conscious experience.



\subsection{Recursive Reality}

Because the simulation replicates the physical laws of our universe, every measurement made by simulated observers within their virtual environment must correspond to the measurements made by real observers in the physical world. To the simulated observers, their universe is perceived as fully real.

Yet we, the real observers, know that these virtual universes are simulated - created via hardware we understand and control.

As virtual observers will discover they can simulate their own existence \cite{bostrom2003} this implies recursion.

Let:

\begin{itemize}
  \item \( \mathcal{S}_i \): the informational state of universe at level \( i \)
  \item \( \mathcal{T}_i \): the execution trace of a simulation running within \( \mathcal{S}_i \)
  \item \( \Phi \): a mapping from an execution trace to a geometric structure (e.g., a simulated universe)
\end{itemize}

We define the simulated universe as:
\[
  \mathcal{S}_{i+1} = \Phi(\mathcal{T}_i)
\]

That is, \( \mathcal{S}_{i+1} \) is the geometric interpretation of the execution trace \( \mathcal{T}_i \) embedded within universe \( \mathcal{S}_i \).

However, given finite set of information, this recursion can not infinite. Every universe \( \mathcal{S}_i \) has finite informational resources. With each successive simulation, entropy loss, quantization, or resolution limits degrade the fidelity of \( \mathcal{T}_i \), so that eventually:

\[
  \mathcal{S}_{i+n} = \bot \quad \text{for some } n \ll \infty
\]

Here, \( \bot \) denotes an informational collapse—an inability to continue the simulation meaningfully due to insufficient entropy or representational capacity.

This formulation supports a finite, information-constrained recursion of simulated universes, each arising as a geometric projection of a computation trace.

If any universe  is virtual—i.e., instantiated from an execution trace within —then by informational symmetry:

If:
\[
  S_{i+1} = \text{Virtual} => S_{i} = \text{Virtual}
\]
Then, by recursion:
\[
  S_i = S_{i+1} = \text{Virtual}
\]

Thus, there is no ontologically privileged level. If even one layer is virtual, they all must be.

\textbf{Conclusion}: Our universe is virtual.



\section{Falsification}

Though full-scale DNA-based simulations are currently infeasible, the model is falsifiable in principle—particularly through the behavioral correlates of subjective phenomena in artificial systems. If a high-fidelity simulation of a human—constructed, for example,
from complete DNA information - were unable to exhibit measurable responses to pain despite demonstrating full functional
equivalence, the fourth assumption of the theory, and thus the model as a whole, would be called into question.



\section{Conclusion}

While Penrose \cite{penrose1989emperor} famously argued that consciousness cannot be captured by any formal axiomatic system , we take the opposite view. We have shown that if four physically grounded assumptions hold - regarding the informational nature of DNA, its obedience to physical law, the universality of computation, and the measurable nature of pain - then humans must necessarily emerge from a computationally formal system.

From this, it follows that consciousness is not tethered to biology or substrate but emerges from information itself. Simulated humans must be conscious and pain-sensitive if they follow the same formal rules. Moreover, subjective time must be an emergent property of an observer’s internal structure—not an external, objective feature of reality.

These findings imply that physical reality itself is informational and observer centric.


\section*{Future Work}


In future work, we aim to develop a theory based on the vision that the deep nature of the universe is abstract. All of physics may be recast as emergent geometry arising from information.

\end{document}

