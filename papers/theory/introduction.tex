
\documentclass[12pt]{article}
\usepackage{amsmath,amssymb}
\usepackage{hyperref}
\usepackage{geometry}
\geometry{margin=1in}

\title{Observer-Centric Informational Model of Reality}
\author{Author Name}
\date{\today}

\begin{document}

\maketitle

\begin{abstract}
      We propose a minimal, self-contained model of reality in which the observer is modeled as a finite, pure state wavefunction whose evolution is governed by unitary transformations that maximize predictive compressibility and memory continuity. Subjective time emerges as an ordering of observer states, and quantum superposition arises naturally as a compression of informational content. We demonstrate how spacetime geometry and expanding cosmology emerge from entropy gradients encoded in the observer's information. Our framework reconciles foundational quantum and gravitational phenomena as dual manifestations of underlying informational structures.
\end{abstract}

\section{Introduction}

The longstanding challenge in fundamental physics is to reconcile quantum mechanics and general relativity within a unified framework. We propose an observer-centric informational model of reality, wherein the observer's state encapsulates all accessible information, and spacetime emerges as a geometric interpretation of informational relationships. Time itself is not a background parameter but an emergent ordering of observer states.

\section{The Informational Universe}

\subsection{Binary Representations of Reality}

We model reality as finite binary strings of length \(n\),
\[
      b \in \{0,1\}^n,
\]
which encode all informational content accessible to an observer.

\subsection{Observers as Substrings}

An observer is defined as a finite subset of bit positions
\[
      P \subseteq \{0, \dots, n-1\},
\]
with observed data
\[
      O = b|_P,
\]
the projection of the universe bitstring onto those positions.

The observer does not exist as a separate entity embedded in the universe, but rather is identified with this informational substring.

\subsection{Pure State Wavefunction Representation}

While the bitstring representation is foundational, it is often more natural to express the observer's state as a \emph{pure state wavefunction} \(\psi\), a vector in a finite-dimensional Hilbert space,
\[
      \psi \in \mathbb{C}^d,
\]
with normalization \(\|\psi\|=1\).

This wavefunction arises as an optimal, compressed encoding of the observer's bitstring, capturing probabilistic correlations and patterns.

\subsection{Duality of Bitstrings and Wavefunctions}

We emphasize a dual description:

\begin{itemize}
      \item \textbf{Bitstrings} provide a concrete, combinatorial informational substrate.
      \item \textbf{Wavefunctions} provide a compressed, computationally efficient, and predictive model of the observer's information.
\end{itemize}

This duality allows us to shift between discrete and continuous representations, facilitating emergence of physical laws from informational principles.

\section{Subjective Time and Wavefunction Evolution}

\subsection{Time as Ensemble Indexing}

We define subjective time for the observer as an \emph{ordering} of successive ensembles or states,
\[
      T = \{ \psi_0, \psi_1, \psi_2, \ldots \},
\]
where each \(\psi_t\) is a pure state representing the observer’s informational content at subjective time \(t\).

No external absolute time is assumed; time emerges as an index labeling the sequence of observer states.

\subsection{Unitary Evolution as Predictive Compression}

The observer's transition from \(\psi_t\) to \(\psi_{t+1}\) is modeled by a unitary operator \(U\):
\[
      \psi_{t+1} = U \psi_t,
\]
where \(U\) is chosen to maximize predictive compressibility and consistency with prior information.

This reflects the observer’s internal extrapolation of their informational state, maintaining structural continuity over subjective time.

\subsection{Memory and Structural Continuity}

Continuity of experience requires:
\[
      \mathrm{Sim}(\psi_t, \psi_{t+1}) \geq \epsilon,
\]
for some \(\epsilon > 0\), ensuring the observer retains recognizable memory traces and coherent identity.

\section{Compression, Superposition, and Emergent Physics}

\subsection{Compression as a Selection Principle}

We propose that universes supporting many observer trajectories with high continuity have low algorithmic complexity (high compressibility).

This leads to a selection principle favoring:
\[
      \min_{U} C(U),
\]
where \(C(U)\) is the Kolmogorov complexity of universe \(U\).

\subsection{Wavefunction Superposition}

The wavefunction \(\psi\) naturally encodes multiple possible futures as a superposition:
\[
      \psi = \sum_i c_i \phi_i,
\]
with coefficients \(c_i\) reflecting compressibility-weighted likelihoods.

Interference arises due to shared structure among \(\phi_i\), explaining quantum phenomena as computational compression artifacts.

\subsection{Emergent Geometry and Dynamics}

Spacetime geometry, locality, and causality emerge as regularities in the wavefunction's structure, governed by the observer’s internal extrapolation.

Increasing entropy corresponds to increasing complexity in the observer's wavefunction ensemble, which translates into an expanding geometric universe.

\section{Discussion and Outlook}

Our informational framework naturally predicts that the universe's expansion corresponds to increasing entropy, and that the initial state was exactly smooth, encoding zero entropy as a minimal pure state wavefunction. Over subjective time, structure and complexity emerge via compressive evolution and wavefunction superposition.

Future work includes embedding simulations and graphics illustrating the correspondence between informational entropy and geometric expansion, and exploring empirical tests of this minimal observer-centric model.

\end{document}
