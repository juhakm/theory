\documentclass[11pt]{article}
\usepackage{amsmath, amssymb, hyperref}
\usepackage{amsthm}
\usepackage{geometry}
\usepackage[backend=biber]{biblatex}
\addbibresource{../references.bib}
\newtheorem{lemma}{Lemma}
\geometry{margin=1in}
\title{\LARGE A Theory of Information and the Universe}
\author{Juha Meskanen}
\date{2019 -- \today}

\begin{document}

\maketitle

\section{Introduction}

Physics has long grappled with the apparent disjunction between quantum mechanics and general relativity. While the former describes the probabilistic behavior of subatomic particles, the latter governs the smooth, deterministic curvature of spacetime. Unifying these frameworks has proven notoriously difficult, particularly due to the fundamentally different assumptions about locality, determinism, and measurement.

As intelligent observers embedded in the universe, we attempt to model the very system we are part of. This raises foundational limits reminiscent of Gödel's incompleteness theorem: just as no sufficiently powerful formal system can prove all truths about itself, perhaps no internal observer can derive a complete physical theory of the universe from within \cite{dyson2004godel}. These self-referential limits motivate the search for a new foundation grounded not in geometry or fields, but in information.

In this work, we reject the presumption that spacetime or particles are ontologically fundamental \cite{wheeler1990it} \citation{zuse1970calculating}. We show that the universe is made of abstract information, provided only four observable assumptions hold. Spacetime and particles must then be emergent properties arising from information \cite{zuse1970calculating} \cite{tegmark2008mathematical}. Our goal is to develop a framework in which everything we perceive—including geometry, matter, time, and even consciousness—can be derived from information.

\section{Four Observational Assumptions}

We begin with four assumptions that are not metaphysical postulates, but observationally grounded regularities evident in both physics and computation. These assumptions are minimal, falsifiable, and intentionally free of ontological commitments. From them, we derive a striking conclusion: the universe must be fundamentally informational, and all informational states that satisfy these assumptions already exist.

This line of reasoning allows us to reinterpret major physical features as emergent informational phenomena. It provides unified explanations for long-standing questions such as:
\begin{itemize}
      \item Why time flows and why it has a direction.
      \item What governs the structure of black hole singularities.
      \item How spacetime geometry can emerge from discrete informational patterns.
      \item Whether a minimal model of the universe can generate the laws of physics.
\end{itemize}

Each of these topics is addressed using formal reasoning and supported by mathematical models and Python simulations.

The framework developed here assumes no primitive physical substrate—no predefined spacetime, particles, or fields. Everything we observe, from matter to consciousness, is modeled as an emergent structure within a fundamentally informational universe.


\end{document}
