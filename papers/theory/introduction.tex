\documentclass[11pt]{article}
\usepackage{amsmath, amssymb, hyperref}
\usepackage{amsthm}
\newtheorem{lemma}{Lemma}
\usepackage{geometry}
\geometry{margin=1in}
\title{\LARGE A Theory of Information and the Universe}
\author{Juha Meskanen}
\date{2019 -- \today}

\begin{document}


\maketitle

\section{Introduction}

Physics has long grappled with the apparent disjunction between quantum mechanics and general relativity. While the former describes the probabilistic behavior of subatomic particles, the latter governs the smooth, deterministic curvature of spacetime. Unifying these frameworks has proven notoriously difficult, particularly due to the fundamentally different assumptions about locality, determinism, and measurement.

We, intelligent beings, also observe the universe from within, which makes us part of the system we are trying to understand. This may prevent us from developing a complete theory due to Gödel's incompleteness theorem. Furthermore, no theory can be regarded as a complete theory of everything if it does not include the conscious and pain-sensitive observer.

In this work, we reject the presumption that spacetime or particles are ontologically fundamental. We show that the universe is made of abstract information, provided only four observable assumptions hold. Spacetime and particles must then be emergent properties arising from information. Our goal is to develop a framework in which everything we perceive—including geometry, matter, time, and even consciousness—can be derived from information.

\section{Four Informational Assumptions}

We begin with four assumptions that are observationally and logically plausible. From these assumptions, we derive a powerful conclusion: the universe must be fundamentally informational in nature, and all possible informational states already exist.

In this framework:

\begin{itemize}
      \item \textbf{Matter and fields} are geometric projections of information that correlate with the observer's information.
      \item \textbf{Time} is an emergent property that arises when observers find themselves in the most probable configurations.
      \item \textbf{Gravity} is not a force but a statistical tendency to move along entropy-increasing trajectories.
      \item \textbf{Consciousness} is a statistical filter.
\end{itemize}

By developing this line of reasoning, we will reinterpret major physical concepts as informational phenomena and resolve long-standing paradoxes such as:

\begin{itemize}
      \item The nature of time and its arrow.
      \item Physics governing black hole singularities.
      \item The emergence of spacetime from discrete informational states.
      \item A minimal model of the universe with physical laws as emergent properties.
\end{itemize}

Each of these topics will be demonstrated and backed by mathematical models and simulations written in Python.

The framework offers a minimal, computable, and logically consistent foundation for physics. It assumes no physical predefined substrates besides information. The existence of observers, measurement, gravity, and time is expressed within a single unifying theory.

What follows is not speculation, but deduction. If the four initial axioms hold, the universe must be an abstract informational object.

\end{document}
