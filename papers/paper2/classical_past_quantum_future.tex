
\documentclass{article}
\usepackage{amsmath, amssymb, graphicx, hyperref}

\title{Simulation of Observer-Centric Wavefunction Evolution}
\author{Juha Meskanen}
\date{\today}

\begin{document}

\maketitle

\section{Introduction}
This document serves as a supplementary explanation of the computational model used to simulate the evolution of an observer-centric wavefunction $\Psi(x, y, t)$ within a discretized environment. The simulation aims to illustrate a physically motivated separation between a \emph{collapsed classical past} and a \emph{superposition-based quantum future}.

\section{Conceptual Foundation}

The observer is treated as a part of the universal wavefunction, encoded as a bistring (binary string) that evolves in discrete time steps. At each time step, the simulation records a snapshot of the observer's knowledge of their universe, saved as images named \texttt{observer*.png}. These images visually represent $\Psi(x, y, t)$ as perceived by the observer at that moment.

The simulation embodies the principle that information from the environment is encoded into the observer's state. Consequently, the classical past---where observations have already been made---is depicted with well-defined, collapsed features, while the quantum future remains open, represented by interference and probabilistic behavior.

\section{Technical Description}

The simulation discretizes time, space, and wavefunction amplitude. At each iteration:
\begin{enumerate}
    \item The current observer state is updated using a complex-valued wavefunction $\Psi(x, y, t)$.
    \item The amplitude and phase are stored for visualization and analysis.
    \item An optimization pass can be applied to reconstruct or approximate future states using a Fourier basis, but this is an optional enhancement.
\end{enumerate}

\section{Mathematical Model}

We define the wavefunction $\Psi(x, y, t)$ over a grid of size $H \times W$ with $T$ discrete time steps. The evolution may be governed by Schrödinger-type dynamics or any other physically inspired rule.

\subsection{Fourier-Based Approximation (Optional)}
An optional optimization pass fits a linear combination of temporal basis functions (e.g., $\sin(\omega t), \cos(\omega t)$) to approximate $\Psi$ over time. This is computationally efficient but sacrifices the complex-valued interference patterns crucial for simulating full quantum behavior.

\section{Usage and Output}

The simulation outputs image files representing the wavefunction at each time step from the observer's point of view. These snapshots can be compiled into a video or animation, enabling visual inspection of phenomena such as interference, decoherence, and the gradual divergence between the classical past and quantum future.

\section{Conclusion}

This simulation framework visualizes the flow of time from an observer-centric viewpoint within a discretized quantum universe. The focus on bistring evolution and wavefunction imagery provides a novel approach to understanding how localized observation leads to classicality, while future possibilities remain open in superposition.

\end{document}
