\documentclass[11pt]{article}
\usepackage{amsmath,amsfonts,amssymb}
\usepackage{graphicx}
\usepackage{hyperref}
\usepackage{authblk}
\usepackage[backend=biber]{biblatex}
\newtheorem{lemma}{Lemma}
\newtheorem{definition}{Definition}
\newtheorem{remark}{Remark}

\addbibresource{../references.bib}
\title{The Abstract Universe: A Minimal Informational Model of Reality}
\author{Juha Meskanen}
\date{July 2025}

\begin{document}

\maketitle


\begin{abstract}
    Building on our earlier proposal that the universe is fundamentally informational in nature,
    and that the laws of physics emerge due to observer centric constraints, we propose a radically minimalist foundation for physics based on information only.
    The framework assumes no primitive physical substrate—no predefined spacetime, particles, or fields.
    Everything we observe, from matter to consciousness, is modeled as an emergent structure within a fundamentally informational universe.
    Spacetime itself arises from the internal extrapolation of the observer’s finite wavefunction.
\end{abstract}

\section{A Minimal Informational Model of Reality}

This section describes a minimal model of reality based on the hypothesis that observers are substrings of bitstrings. No further structure is assumed.
The observer does not exist within spacetime — rather, its internal structure defines the scope and scale of what spacetime is.

\subsection{Ontological Basis}

We assume:

\begin{enumerate}
    \item Reality consists of binary strings of length $n$: $b \in \{0,1\}^n$.
    \item An observer is a subset of positions $P \subseteq \{0,1,\dots,n-1\}$.
    \item Observed reality is defined by selecting those positions: $O = b|_P$.
\end{enumerate}

No concept of time, matter, or causality is assumed beyond this.



\subsection{Subjective Time as Emergent Ordering}

The illusion of time is a natural result of the observer’s internal state evolving from one bitstring configuration to the next.


Specifically, each observer defines a sequence
\[
    P = (i_0, i_1, \dots, i_k),
\]
where \(i_j \in \{0,\dots,n-1\}\). This sequence imposes a \emph{structural} temporal axis—an emergent phenomenon based entirely on the ordering of positions—without invoking any active or external process.

\subsection{Formal Definition of Observer Memory via Structural Similarity}

Let \(U = (b_0,b_1,\dots,b_{N-1})\). An observer is defined by a subset \(P\subseteq \{0,\dots,N-1\}\) and derives bits \(O = U|_P\).

We introduce \emph{memory} as follows:

\begin{definition}[Memory via Similarity]
    An observer exhibits memory if there exist non-empty subsets \(M\subseteq P\) and \(S\subseteq \{0,\dots,N-1\}\) such that
    \[
        \mathrm{Sim}(U|_M, U|_S)\;\ge\;\theta,
    \]
    for a threshold \(\theta\in[0,1]\).
\end{definition}

Here, \(\mathrm{Sim}\) is a similarity function. For equal-length bitstrings:
\[
    \mathrm{Sim}(x,y)=1-\frac1{|x|}\sum_{j=1}^{|x|} |x_j-y_j|.
\]
If \(x,y\) differ in length, interpretations include truncating or extending the shorter string with a neutral value.

\subsection{Probability and Observer Prevalence}

We define probability as a relative frequency over compatible observers:

\begin{definition}[Observer Probability]
    Let $S$ be the set of all observers compatible with universe $U$. Then the probability of $O$ is
    \[
        \mathbb{P}(O) = \frac{|\{O' \in S : O' \sim O\}|}{|S|},
    \]
    where $\sim$ denotes structural similarity above a fixed threshold.
\end{definition}

\subsection{Observer Continuity Lemma (OCL)}

Let \(O=(o_1,\dots,o_n)\) be an observer’s internal trajectory and \(F=(f_1,\dots,f_n)\) be corresponding universe-frame bitstrings. We require:

\[
    \exists\;\epsilon>0\;\text{such that}\;\mathrm{Sim}(o_i, f_i)\ge\epsilon\;\text{for all }i.
\]

This condition ensures structural continuity and provides the formal basis for “memory” and coherent subjective experience without appealing to causal interactions.

\subsection{Observer-Centric Universe Compression Principle (OCUCP)}

We propose the following principle:

\begin{quote}
    The most probable universes are those that maximize the number of compatible observer trajectories under structural continuity constraints.
\end{quote}

Formally, let $U$ be a candidate universe, and $T(U)$ the set of observer trajectories satisfying the OCL. Then:

\[
    \mathbb{P}(U) \propto |T(U)|.
\]

This leads to a selection bias toward simple, compressible universes with many observers.

\subsection{Compression and the Wavefunction}

\begin{definition}[Compression Bias Principle (CBP)]
    Let \(F\) be a sequence of universe frames. Let \(C(F)\) denote its compressed bit-length. Then:
    \[
        \#\{\text{observer trajectories embedded in }F\}\;\propto\; \frac{1}{C(F)}.
    \]
\end{definition}

From this emerges a model of the quantum-like wavefunction:

\begin{itemize}
    \item The observer’s internal model maintains past states \(\{o_1,\dots,o_t\}\).
    \item Potential future continuations \(\{o_{t+1},\dots,o_{t+k}\}\) are extrapolated via compressive pattern completion.
    \item If the compression basis is Fourier/sinusoidal, the predicted frames are inherently smooth and oscillatory.
    \item Multiple continuations form a \emph{superposition}, weighted by compressibility, with interference patterns arising from shared substructures.
\end{itemize}

Under this interpretation, the wavefunction is not a physical field but a \textbf{computationally optimal compression model} representing the observer.

Instead of branching \cite{everett1957relative} this model implies that universes - all the different configurations of bitstrings - exist simulaneously.


\subsection{Entanglement and Memory Sharing}

Let two observers \(O_1, O_2\) share a common prefix \(S\). Then:

\begin{definition}[Entangled Observers]
    Observers \(O_1\) and \(O_2\) are entangled if \(\mathrm{Sim}(O_1|_S, O_2|_S)\ge\theta\) for some threshold \(\theta\).
\end{definition}

Such observers exhibit correlated outcomes without communication, due to shared structural ancestry.

Crucially, we reinterpret the wavefunction as not embedded within a preexisting spacetime. Instead, the wavefunction defines its own coordinate system: the axes of space and time are abstract dimensions that emerge from the structure of the wavefunction itself. This means that the observer, encoded as a compressed set of oscillatory modes, defines the possible futures and the geometry of its surroundings via extrapolation.


\section{Gravity}




\section{Discussion and Implications}

\subsection{Finite Information}

We do \emph{not assume} that reality must be finite in absolute terms. Instead, finiteness follows from two operational principles:
\begin{enumerate}
    \item Conservation of information implies bounded total informational content.
    \item An observer can \emph{only} access and encode a finite subset of information.
\end{enumerate}
Infinite structures remain mathematically conceivable but play no role in the observer’s emergent perception.

\subsection{Observer as Wavefunction}

We propose the quantum wavefunction arises from the \textbf{most efficient compression scheme} a finite observer can construct. Periodic functions—especially sine waves—are favored due to their optimal information density and simplicity, consistent with empirical smoothness in physical laws.

The wavefunction encodes all the information in the observer past and the present. Future is the information extrapolated with the wavefunction.


\subsection{Empirical Implications}

While abstract, this framework generates testable implications:

\begin{itemize}
    \item The universe should exhibit compressible, recursive structure.
    \item High-entropy or uncorrelated states (e.g. pure noise) should rarely support coherent observers.
    \item Apparent laws of physics reflect compressibility bias, not ontological necessity.
\end{itemize}


\section{Supplementary Material: Simulation}

We present a simulation program implementing the formal model as a Python script.

This simulation models an observer defined as a finite length bistring. It uses a compressed representation of the bitstring via Fourier basis functions and identifies regions where observer pattern is likely to occur next by extrapolating the wavefunction.


The program is available at:
\[
    \texttt{http://github.com/juhakm/simulations/paper2/simulation.py}
\]

\end{document}
